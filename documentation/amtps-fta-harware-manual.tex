\documentclass{article}
\usepackage[utf8]{inputenc}

\title{AMPTS FTA Avionics Design}
\author{Matt Ruffner}
\date{\today\\v1.2}


\usepackage{url}
\usepackage{array}
\usepackage{float}
\usepackage{natbib}
\usepackage{graphicx}
\usepackage{listings}
\usepackage{fullpage}
\usepackage{hyperref}
\hypersetup{
    colorlinks=true,
    linkcolor=blue,
    filecolor=magenta,      
    urlcolor=cyan,
}

\newcommand{\ddrev}{E}

\begin{document}


\maketitle
\begin{center}
	%\includegraphics[width=4cm]{images/ukcoe.png}
	%\includegraphics[width=4cm]{images/upaero-logo.png}
	%\includegraphics[width=2cm]{images/nasa-logo.png}
	%\includegraphics[width=4cm]{images/ornl.pdf}	
\end{center}
\tableofcontents \newpage
\listoffigures 
\listoftables \newpage



%%%%%%%%%%%%%%%%%%%%%%%%%%%%%%%%%%%%%%%%%%%%%%%%%%%%%%%%%%%%%%%%%%%%%%%%%%%%%%%%%%%%%%%%%
%%%%%%%%%%%%%%%%%%%%%%%%%%%%%%%%%%%%%%%%%%%%%%%%%%%%%%%%%%%%%%%%%%%%%%%%%%%%%%%%%%%%%%%%%
%%%%%%%%%%%%%%%%%%%%%%%%%%%%%%%%%%%%%%%%%%%%%%%%%%%%%%%%%%%%%%%%%%%%%%%%%%%%%%%%%%%%%%%%%
%%%%%%%%%%%%%%%%%%%%%%%%%%%%%%%%%%%%%%%%%%%%%%%%%%%%%%%%%%%%%%%%%%%%%%%%%%%%%%%%%%%%%%%%%
\section{Introduction}
This document outlines the initial design of electrical hardware that will support the flight test article (FTA) being developed for the NASA Flight Opportunities \textit{Additive Manufacturing of Thermal Protection Systems} (AMTPS) project. This project is a collaboration between several NASA facilities including JSC and Langley as well as other research institutions namely the University of Kentucky and Oak Ridge National Laboratories. Figure \ref{fig:capsule-overview} shows a general overview of the various components that are  expected to be a part of the flight test article (referred to as 'the capsule') based on current requirements.

Capsule hardware must be capable of robust data acquisition in order to characterize the thermal, pressure, and overall heating environment of supersonic reentry. In addition to collecting multiple data channels, the hardware must be capable of transmitting its location as it touches back down to Earth. The multichannel data acquisition system must also save flight data for further analysis after it touches back down. Timing the deployment of a parachute before touchdown is also a requirement of the hardware in order to lessen the force of impact.

This document is organized as follows: Section \ref{sec:requirements} discusses needed capabilities in the form of technical and logistical requirements set by the NASA project design document (rev. \ddrev ). Section \ref{sec:ss-design} displays the preliminary designs and testing hardware for the various subsystems of the capsule. This includes data acquisition, logging, and vehicle-to-ground communication for recovery operations. Section \ref{sec:logistics} contains in-work project considerations as well as an approximate schedule. Appendix \ref{appa} contains schematics for PCBs that have been prototyped. Appendix \ref{appb} contains images of testing hardware that has been developed along with initial experimental results. 

\begin{figure}[h!]
	\centering
	\includegraphics[width=\textwidth]{images/amtps-avionics.png}
	\caption{Capsule component overview, showing the various functional blocks.}
	\label{fig:capsule-overview}
\end{figure}


%%%%%%%%%%%%%%%%%%%%%%%%%%%%%%%%%%%%%%%%%%%%%%%%%%%%%%%%%%%%%%%%%%%%%%%%%%%%%
%%%%%%%%%%%%%%%%%%%%%%%%%%%%%%%%%%%%%%%%%%%%%%%%%%%%%%%%%%%%%%%%%%%%%%%%%%%%%
%%%%%%%%%%%%%%%%%%%%%%%%%%%%%%%%  APPENDIX
%%%%%%%%%%%%%%%%%%%%%%%%%%%%%%%%%%%%%%%%%%%%%%%%%%%%%%%%%%%%%%%%%%%%%%%%%%%%%
%%%%%%%%%%%%%%%%%%%%%%%%%%%%%%%%%%%%%%%%%%%%%%%%%%%%%%%%%%%%%%%%%%%%%%%%%%%%%
\section{Requirements}
\label{sec:requirements}
Based off of the most recent revision (\ddrev) of the flight test requirements document.

\begin{enumerate}
\item Instrumentation and telemetry
  \begin{enumerate}
	  \item Shall support betwen 8 and 20 thermocouples of varying type
	  \item Shall support up to 6 absolute pressure sensors
	  \item Shall support at least 1 intertial measurement unit (IMU)
	  \item Should support 1 heat flux sensor
	  \item Shall contain a GPS for recovery operations, accurate to with 100m
	  \item Capsule shall contain an internal barometric pressure sensor 
	  \item Telemetry data shall be collected at a minumum of 10Hz
	  \item Telemetry data shall be stored to onboard nonvolatile memory that will survive landing
	  \item Location telemetry shall be transmitted through a vehicle-to-ground system (e.g. Iridium satellite, Xbee)
	  \item Recovery location should be broadcasted at least once every 5 minutes post-flight
  \end{enumerate}
\item Activation and flight sequencing
  \begin{enumerate}
	  \item Shall be powered through the duration of the flight
	  \item Shall support continuous operation between -20 deg C and 80 deg C
	  \item Shall support pre-launch activation on the ground; should support low power mode prior to deployment
	  \item Shall detect and/or sense when deployment has occurred via interfacing with the launch vehicle
	  \item Shall transmit in-flight telemetry with position information
	  \item In-flight telemetry should contain capsule velocity 
	  \item Shall trigger parachute deployment at a specified time
  \end{enumerate}
\item Physical properties
\item Avionics hardware shall weigh under or around 0.5kg
\item Shall cost under \$3,000
\end{enumerate}




%%%%%%%%%%%%%%%%%%%%%%%%%%%%%%%%%%%%%%%%%%%%%%%%%%%%%%%%%%%%%%%%%%%%%%%%%%%%%
%%%%%%%%%%%%%%%%%%%%%%%%%%%%%%%%%%%%%%%%%%%%%%%%%%%%%%%%%%%%%%%%%%%%%%%%%%%%%
%%%%%%%%%%%%%%%%%%%%%%%%%%%%%%%%  DESIGN
%%%%%%%%%%%%%%%%%%%%%%%%%%%%%%%%%%%%%%%%%%%%%%%%%%%%%%%%%%%%%%%%%%%%%%%%%%%%%
%%%%%%%%%%%%%%%%%%%%%%%%%%%%%%%%%%%%%%%%%%%%%%%%%%%%%%%%%%%%%%%%%%%%%%%%%%%%%
\section{Subsystem Design}
\label{sec:ss-design}

\subsection{Main command and data handling}

Using off the shelf processors for convenience, and also to avoid bottlenecks in prototyping due to unpredictable chip shortages. Currently, two development boards from Adafruit are being used to prototyping and are listed below in Table \ref{tab:processors}\footnote{TPM: Temperature and Pressure Measurement}\footnote{CDH: Command and Data Handling}. The diagram in Figure \ref{fig:main-overview} provides a general overview of the organization of electronic hardware within the capsule. 

\begin{figure}[h!]
	\centering
	\includegraphics[width=\textwidth]{images/amtps-main-system.png}
	\caption{Main functional components of the capsule and the communication buses between them.}
	\label{fig:main-overview}
\end{figure}


\begin{table}[h!]	
	\caption{List of processors used and their capabilities.}
	\begin{tabular}{l | c c m{5cm}}
		Part & Description & Role & Product Link \\
		\hline
		(2) Feather M0 Basic Proto & Cortex-M0 @ 48 MHz & TPM and IMU processor & \url{https://www.adafruit.com/product/2772}\\
		Feather M4 Express & Cortex-M4 @ 120Mhz & CDH Processor & \url{https://www.adafruit.com/product/3857} 
	\end{tabular}
	\label{tab:processors}
\end{table}


%%%%%%%%%%%%%%%%%%%%%%%%%%%%%%%%%%%%%%%%%%%%%%%%%%%%%%%%%%%%%%%%%%%%%%%%%%%%%
%%%%%%%%%%%%%%%%%%%%%%%%%%%%%%%%%%%%%%%%%%%%%%%%%%%%%%%%%%%%%%%%%%%%%%%%%%%%%
\subsection{Temperature and Pressure Measurement}
The temperature and pressure measurement (TPM) Subsystem (TPMS) supports collection of several datapoints needed for recreating the heating environment of reentry. The diagram in Figure \ref{fig:tpms-overview} shows an overview of the temperature and pressure monitoring subsystem. Images of prototyping the TPMS are shown in  Appendix \ref{appb}.

In order to support the large number of TCs required in the capsule, a breakout board supporting 6 of the MCP9600T-E/MX series TC to digital converter was designed. This TC converter chip was selected due to its support for a wide variety of TC types (K, J, T, N, S, E, B and R)\footnote{\url{https://www.digikey.com/en/products/detail/microchip-technology/MCP96L00T-E-MX/9606988}}, as well as its chainable I$^2$C interface, allowing more sensing elements to connected with less wiring (similar SPI based conversion chips have a chip select line per chip that would restrict the number of pins available for other capsule functionality).


For pressure meausrement, Honeywell sensing solutions temperature compensated absolute digital pressure sensors were selected and are shown in Table \ref{tab:pressure-sensors}. These single port absolute pressure sensors are available with a variety of sensitivities, allowing a different sensor to be exchanged later on in the design process if it is determined that the current predicted pressure environment is no longer accurate.


\begin{table}[h!]
	\caption{List of pressure sensors and their capabilities}
	\begin{tabular}{c | c m{9cm}}
		Part & Measurement Range & Product Link \\
		\hline
		SSCSRNN015PA3A3 & 0-103.42 kPa & \url{https://www.digikey.com/en/products/detail/honeywell-sensing-and-productivity-solutions/SSCSRNN015PA3A3/2416212}\\
		SSCSRNN1-6BA7A3 & 0-160 kPa & \url{https://www.digikey.com/en/products/detail/honeywell-sensing-and-productivity-solutions/SSCSRNN1-6BA7A3/2416214} 
	\end{tabular}
\label{tab:pressure-sensors}
\end{table}

\begin{figure}[h!]
	\centering
	\includegraphics[width=\textwidth]{images/amtps-temp-pressure-subsystem.png}
	\caption{Functional decomposition of the TPMS.}
	\label{fig:tpms-overview}
\end{figure}


\subsection{Inertial Measurement}
A breakout board for the H3LIS100DL~\footnote{\url{https://www.digikey.com/en/products/detail/stmicroelectronics/H3LIS100DL/7313278}} +/- 100g accelerometer has been designed to help in the evaluation of this IC as an effective way to measure acceleration loads on the capsule. The ADXL377 is another potential high-g 3-axis acceleromter~\footnote{\url{https://www.adafruit.com/product/1413}}, but has analog output and higher cost since it is currently only available in a pre-assembled breakout board.

In addition to a high-g 3-axis accelerometer, a lower dynamic range 6 axis acceleromter+gyroscope chip will also be included in the capsule (such as a ICM-20949 or MPU6050).

%%%%%%%%%%%%%%%%%%%%%%%%%%%%%%%%%%%%%%%%%%%%%%%%%%%%%%%%%%%%%%%%%%%%%%%%%%%%%
%%%%%%%%%%%%%%%%%%%%%%%%%%%%%%%%%%%%%%%%%%%%%%%%%%%%%%%%%%%%%%%%%%%%%%%%%%%%%
\subsection{Telemetry}

Iridium modem to send GPS coordinates for recovery operations. University of Kentucky has an Iridium modem available to use, with an account providing message credits. Discussion on the use of this modem is yet to have happened. Additional vehicle-to-ground telemetry would be reassuring as the Iridium network can be unreliable for short periods of time. It is unclear whether this would be achievable with a COTS radio link (such as XBee or LORA) that would not require FCC certification. This requires coordination with launch site for tracking with a directional antenna and other hardware setup. The appeal of the Iridium is we get an email with the GPS coordinates of the capsule and we can tell the recovery crew the location of the capsule from anywhere. 


%%%%%%%%%%%%%%%%%%%%%%%%%%%%%%%%%%%%%%%%%%%%%%%%%%%%%%%%%%%%%%%%%%%%%%%%%%%%%
%%%%%%%%%%%%%%%%%%%%%%%%%%%%%%%%%%%%%%%%%%%%%%%%%%%%%%%%%%%%%%%%%%%%%%%%%%%%%
\subsection{Parachute Deployment}
Main parachute deployment will use a CO$_2$ cartridge puncture system which has a servo actuated trigger. Servo control signal pins are allotted on the main CDH processor and configured in software to trigger when necessary. Current plans to decide correct deployment time are to use capsule internal barometric pressure, possibly in combination with one of the external pressure sensors.


%%%%%%%%%%%%%%%%%%%%%%%%%%%%%%%%%%%%%%%%%%%%%%%%%%%%%%%%%%%%%%%%%%%%%%%%%%%%%
%%%%%%%%%%%%%%%%%%%%%%%%%%%%%%%%%%%%%%%%%%%%%%%%%%%%%%%%%%%%%%%%%%%%%%%%%%%%%
\subsection{Storage}
On board non-volatile storage is required to log in-flight telemetry data for post processing. Due to the high vibrational loads expected during launch, an SD card might be unreliable (spring loaded contacts could separate from card). For this reason, solid state flash integrated circuits are being explored that can store just as much information as an SD card. Exporting logged data from the capsule would then be done via cabled connection to a computer after capsule recovery.


%%%%%%%%%%%%%%%%%%%%%%%%%%%%%%%%%%%%%%%%%%%%%%%%%%%%%%%%%%%%%%%%%%%%%%%%%%%%%
%%%%%%%%%%%%%%%%%%%%%%%%%%%%%%%%%%%%%%%%%%%%%%%%%%%%%%%%%%%%%%%%%%%%%%%%%%%%%
\subsection{Power}
To power the capsule avionics, a lithium polymer battery will be used. Most likely a 2 cell (in series) with a high quality switching step down regulator to provide a 5 volt system bus. A single cell approach (with no 5v regulator) eases charging but might cause system electronics to be unreliable during low battery voltage scenarios. No matter what cell configuration is chosen, a battery protection IC will be used for added safety from shorts and battery depletion.


%%%%%%%%%%%%%%%%%%%%%%%%%%%%%%%%%%%%%%%%%%%%%%%%%%%%%%%%%%%%%%%%%%%%%%%%%%%%%
%%%%%%%%%%%%%%%%%%%%%%%%%%%%%%%%%%%%%%%%%%%%%%%%%%%%%%%%%%%%%%%%%%%%%%%%%%%%%
\subsection{Durability}
Syntactic foam may be used. Consisting of glass microballoons with epoxy resin, it can encase electronics to protect from very high acceleration loads during ascent. Currently working with NASA JSC to get glass microballoons to UK for testing.


%%%%%%%%%%%%%%%%%%%%%%%%%%%%%%%%%%%%%%%%%%%%%%%%%%%%%%%%%%%%%%%%%%%%%%%%%%%%%
%%%%%%%%%%%%%%%%%%%%%%%%%%%%%%%%%%%%%%%%%%%%%%%%%%%%%%%%%%%%%%%%%%%%%%%%%%%%%
\section{Subsystem Software}
\label{sec:software}

Capsule operation sequence is planned as follows:

\begin{enumerate}
	\item Low power state prior to launch
	\item Activate main operation loop upon separation from launch vehicle
	\item Begin logging all telemetry to internal storage and periodically transmitting Iridium packets with position and velocity information for recovery operations
	\item At a specified capsule internal barometric pressure, trigger the CO$_2$ release servo to release the drogue and main parachute.
\end{enumerate}

Currently, verified functionalities include logging of accelerometer, pressure, thermocouple, and GPS data to internal storage. Next steps include adding sending of Iridium modem packets with current position and velocity as well as integrating the IMU logging functionality (currently waiting on parts to be delivered).

Software for the capsule is under version control at \url{http://github.com/krups/amtps-fta-software}. This is a private repository; if anyone would like access, please send an email to matthew.ruffner@uky.edu with their Github username and they will be given access.

%%%%%%%%%%%%%%%%%%%%%%%%%%%%%%%%%%%%%%%%%%%%%%%%%%%%%%%%%%%%%%%%%%%%%%%%%%%%%
%%%%%%%%%%%%%%%%%%%%%%%%%%%%%%%%%%%%%%%%%%%%%%%%%%%%%%%%%%%%%%%%%%%%%%%%%%%%%
\section{Project Logistics}
\label{sec:logistics}

\subsection{Cost}
According the the most recent revision of the project outline, NASA has alotted \$3,000 towards the avionics design. Currently the University of Kentucky has not used any of these funds in the prototyping of capsule hardware.


\subsection{Schedule}
Figure \ref{fig:schedule} shows an approximate schedule for assembly and evaluation of test hardware over the later half of 2021. The project is current slightly ahead of schedule with subsystem testing occurring roughly a month earlier than expected.
\begin{figure}[H]
	\centering
	\includegraphics[width=\textwidth]{images/schedule.png}
	\caption{Approximate schedule for completion of test hardware.}
	\label{fig:schedule}
\end{figure}





%%%%%%%%%%%%%%%%%%%%%%%%%%%%%%%%%%%%%%%%%%%%%%%%%%%%%%%%%%%%%%%%%%%%%%%%%%%%%
%%%%%%%%%%%%%%%%%%%%%%%%%%%%%%%%%%%%%%%%%%%%%%%%%%%%%%%%%%%%%%%%%%%%%%%%%%%%%
%%%%%%%%%%%%%%%%%%%%%%%%%%%%%%%%  APPENDIX
%%%%%%%%%%%%%%%%%%%%%%%%%%%%%%%%%%%%%%%%%%%%%%%%%%%%%%%%%%%%%%%%%%%%%%%%%%%%%
%%%%%%%%%%%%%%%%%%%%%%%%%%%%%%%%%%%%%%%%%%%%%%%%%%%%%%%%%%%%%%%%%%%%%%%%%%%%%
\appendix


\section{Schematics}
\label{appa}


\subsection{Pre-made processor breakout boards}
For the TPM board, CDH board, and IMU logger board, existing processor breakouts will be used due to shortages of bare chips.
\begin{figure}[H]
	\centering
	\includegraphics[width=\textwidth]{images/adafruit_featherm4_schematic}
	\caption{Feather M4 Express schematic, used in the main CDH board.}
	\label{fig:schematic-featherm4}
\end{figure}
\begin{figure}[H]
	\centering
	\includegraphics[width=\textwidth]{images/adafruit_featherm0_schematic}
	\caption{Feather M0 schematic, used in the IMU logger and TPM boards.}
	\label{fig:schematic-featherm0}
\end{figure}



\newpage
\subsection{TC to Digital Schematics}
Figures \ref{fig:schematic-tc-breakout-p1} and \ref{fig:schematic-tc-breakout-p2} show the schematics of the 6 channel TC to digital converter breakout board.
\begin{figure}[H]
	\centering
	\includegraphics[width=\textwidth]{images/tc-breakout-p1}
	\caption{TC to digital schematics, page 1/2}
	\label{fig:schematic-tc-breakout-p1}
\end{figure}

\begin{figure}[H]
	\centering
	\includegraphics[width=\textwidth]{images/tc-breakout-p2}
	\caption{TC to digital schematics, page 2/2}
	\label{fig:schematic-tc-breakout-p2}
\end{figure}



\newpage
\subsection{Main CDH Schematics}
Figures \ref{fig:schematic-cdh-p1} and \ref{fig:schematic-cdh-p2} show the schematics of the main CDH PCB.
\begin{figure}[H]
	\centering
	\includegraphics[width=\textwidth]{images/main_board_schem_1}
	\caption{Main CDH schematics, page 1/2}
	\label{fig:schematic-cdh-p1}
\end{figure}
\begin{figure}[H]
	\centering
	\includegraphics[width=\textwidth]{images/main_board_schem_2}
	\caption{Main CDH schematics, page 2/2}
	\label{fig:schematic-cdh-p2}
\end{figure}


\newpage
\subsection{TPM Schematics}
Figures \ref{fig:schematic-tpm-p1} and \ref{fig:schematic-tpm-p2} show the schematics of the TPM subsystem PCB.
\begin{figure}[H]
	\centering
	\includegraphics[width=\textwidth]{images/tpms_breakout_schem_1}
	\caption{TPM schematics, page 1/2}
	\label{fig:schematic-tpm-p1}
\end{figure}
\begin{figure}[H]
	\centering
	\includegraphics[width=\textwidth]{images/tpms_breakout_schem_2}
	\caption{TPM schematics, page 2/2}
	\label{fig:schematic-tpm-p2}
\end{figure}


\newpage
\subsection{IMU Logger Schematics}
Figures \ref{fig:schematic-imu-p1}, \ref{fig:schematic-imu-p2}, and \ref{fig:schematic-imu-p3} show the schematics of the IMU logger PCB.
\begin{figure}[H]
	\centering
	\includegraphics[width=\textwidth]{images/imu_logger_schem_1}
	\caption{IMU logger schematics, page 1/3}
	\label{fig:schematic-imu-p1}
\end{figure}
\begin{figure}[H]
	\centering
	\includegraphics[width=\textwidth]{images/imu_logger_schem_2}
	\caption{IMU logger schematics, page 2/3}
	\label{fig:schematic-imu-p2}
\end{figure}
\begin{figure}[H]
	\centering
	\includegraphics[width=\textwidth]{images/imu_logger_schem_3}
	\caption{Page 3/3}
	\label{fig:schematic-imu-p3}
\end{figure}





\section{Subsystem Testing}
\label{appb}

Testing hardware is assembled and is currently being used for software development. CAD models of capsule avionics have been integrated into the capsule mechanical CAD (with the exception of wiring) to verify physical fit.  

\begin{figure}[h!]
	\includegraphics[width=\textwidth]{images/subsystem-hardware.jpg}
	\caption{Initial build of subsystem testing hardware. Shown here: Top right: 18 temperature and 5 channel TPM subsystem connected to main CDH subsystem (bottom right). SD card logging and GPS receiver and antenna also shown (top left). Bottom left: Iridium satellite modem.}
	\label{fig:subsystem-hardware}
\end{figure}


\begin{figure}[H]
	\centering
	\includegraphics[width=\textwidth]{images/tc-board-bottom}
	\caption{Bottom of V1 Evaluation board for MCP9600 TC to digital converter}
	\label{fig:tc-board-bottom}
\end{figure}

\begin{figure}[h!]
	\centering
	\includegraphics[width=\textwidth]{images/tc-board-top}
	\caption{Top of V1 Evaluation board for MCP9600 TC to digital converter}
	\label{fig:tc-board-top}
\end{figure}

\begin{figure}[h!]
	\centering
	\includegraphics[width=\textwidth]{images/100g-accel-board}
	\caption{High g accelerometer testing PCB.}
	\label{fig:accel-board}
\end{figure}

\begin{figure}[h!]
	\centering
	\includegraphics[width=\textwidth]{images/pressure-sensor-testbed-annotated}
	\caption{Testing setup for the pressure sensors.}
	\label{fig:pressure-testbed}
\end{figure}

\begin{figure}
	\centering
	\includegraphics[width=\textwidth]{images/tvac-20210602-large.png}
	\caption{Pressure readout from 5 Honeywell sensors over an 8 minute session in a thermal vacuum chamber. Not only were the Honeywell sensors all very close to each other, these values closely matched the pressure readout on the TVAC chamber.}
	\label{fig:pressure-tvac}
\end{figure}


%\bibliographystyle{plain}
%\bibliography{references}
\end{document}
